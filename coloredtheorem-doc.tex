\documentclass{article}

\usepackage[margin=1.6in]{geometry}
\usepackage{fancyvrb}

%%%%%%%%%%%%%%%%%%%%%%%%%%%%%%%%%%%%%%%%%%
%% AMSMATH stuff
% \usepackage{amsthm}
% \usepackage{thmtools}
% \usepackage{amsfonts}


%%%%%%%%%%%%%%%%%%%%%%%%%%%%%%%%%%%%%%%%%%
%% TCBTHEOREM stuff
\usepackage{coloredtheorem}


%%%%%%%%%%%%%%%%%%%%%%%%%%%%%%%%%%%%%%%%%%
%% EXAMPLE document
\usepackage{amsfonts}
\usepackage{kantlipsum}
\usepackage{tabularx}
\usepackage{xspace}
\newcommand\tcthrm{\texttt{coloredtheorem}\xspace}

%% ALGPSEUDOCODEX stuff
\usepackage{algorithm}
\usepackage[
  noEnd=false,
  beginComment={$\triangleright$~\scriptsize},
  beginLComment={$\triangleright$~},
  endLComment={},
]{algpseudocodex}

\algnewcommand\Input{\textbf{Input:~}}
\algrenewcommand\Output{\textbf{Output:~}}
\algrenewcommand\algorithmicforall{\textbf{for each}}
\algnewcommand\Vspace{% 
  {\setlength\itemsep{-1ex}\item[]~}
}
\algnewcommand\Let[2]{\State \ensuremath{#1 \gets #2}}
\algnewcommand\And{\ensuremath{\wedge}\xspace}
\algnewcommand\Or{\ensuremath{\vee}\xspace}
\algnewcommand\Not{\ensuremath{\neg}\xspace}

\usepackage[colorlinks,linkcolor=blue!80]{hyperref}

%%%%%%%%%%%%%%%%%%%%%%%%%%%%%%%%%%%%%%%%%%
%% MAIN DOCUMENT
\begin{document}
  \title{The \tcthrm package}
  \author{João M. Lourenço\\\texttt{\small joao.lourenco@fct.unl.pt}}
  \date{March 31, 2024}
  
  \maketitle
  

\section{Introduction}
\label{sec:introduction}

The \tcthrm package is a simple environment, which takes no options, and that allows to write stuff inside boxes from \texttt{tcolorbox}.  So, this packages includes \texttt{tcolorbox} if necessary. You may include \texttt{tcolorbox} with your own favourite options prior to including this package.

Akin to \verb!\newtheorem! from the \verb!amsmath! package, the user should start by defining a new \emph{theorem/box} group and customize its aspect. Each new environment will have its own counter/numnering. Notice that \verb!\label{…}! and \verb!\ref{…}! work as expected.  The is also a command to generate the corresponding \emph{list of …}

\section{Usage}
\label{sec:usage}

\begin{description}
  \item[]{\bfseries\verb+\usepackage{coloredtheorem}+}
  \begin{description}
      \item[]Load the \texttt{coloredtheorem} package.  This package will load \texttt{tcolorbox} if necessary.
  \end{description}
  \item[]{\bfseries\Verb[commandchars=!()]+\!tcbth()newtheorem{<envname>}{<Name>}[<tcolorbox options>]+}
  \begin{description}
      \item[]\verb+<envname>+ is the environment name, e.g., \verb!theorem!.
      \item[]\verb+<Name>+ is the name for new environment being defined, e.g., \emph{Theorem}.
      \item[]\verb+<tcolorbox options>+ options to be passed to the \verb+tcolorbox+ environment to customize the boxes for the given environment (this argument is optional).
  \end{description}
  \item[]{\bfseries\Verb[commandchars=!()]+\begin{!tcbth()<envname>}{<Caption>}[<tcolorbox options>]<Contents>\end{!tcbth()<envname>}+}
  \begin{description}
    \item[]\verb+<envname>+ is the environment name, e.g., \verb+theorem+.
    \item[]\verb+<Caption>+ is the caption/title of the box.
    \item[]\verb+<tcolorbox options>+ options to be passed to the \verb+tcolobox+ environment, which will override the defaults given in \verb+\cthnewtheorem+.
    \item[]\verb+<Contents>+ the contents to by typeset inside the coulored environment.
  \end{description}
  \item[]{\bfseries\Verb[commandchars=!()]+\!tcbth()listof<envname>s+}
  \begin{description}
      \item[]\verb+<envname>+ is the environment name, e.g., e.g., \Verb[commandchars=!()]+\!tcbth()listoftheorems+.  Please notice that there is a `s' (plural) after \verb+<envname>+.
  \end{description}
\end{description}

\section{Example}
\label{sec:examples}

Let's start by creating two new environments, one for \emph{algorithms} and another for \emph{examples}, both defaulting to a gray frame, the former with a yellowish background and the latter with a lighter gray background.

\begin{verbatim}
\cthnewtheorem{algorithm}{Algorithm}[coltitle=black, colback=yellow!10,
                                       colframe=black!15]
\cthnewtheorem{example}{Example}[coltitle=black, colback=black!5,
                                   colframe=black!30]  
\end{verbatim}

\cthnewtheorem{algorithm}{Algorithm}[coltitle=black, colback=yellow!10, colframe=black!15]
\cthnewtheorem{example}{Example}[coltitle=black, colback=black!5, colframe=black!30]


An algorithm box will be created with:
\begin{verbatim}
\begin{cthalgorithm}{Advance a counter to the next value in a domain
                    $\omega \in \mathbb{N}$.}

Algorithm body here!
\end{cthalgorithm}
\end{verbatim}

\newcommand{\bracesemptyset}{\ensuremath{\lbrace\,\rbrace}}

\begin{cthalgorithm}{Advance a counter to the next value in a domain $\omega \in \mathbb{N}$.}
  \label{alg:advance1}
\begin{algorithmic}[1]
  \Statex \Input $\alpha$: The current value; and $\omega$: the domain.
  \Statex \Output The next value of $\alpha$ within the domain $\omega$.
  \LComment{\color{blue}
            e.g., $\alpha = 3 \And \omega = \lbrace 8, 3, 2, 21, 5, 1 \rbrace \rightarrow 5$}

  \Vspace

  \Function{advance}{$\alpha$: int, $\omega$: set} $\rightarrow$ int \textbf{is}
    % \Statex\Comment{As we are incrementing one by one on the operation IDs, the complexity of this function is linear in the value of the largest operation id.}
    \Let {\alpha} {\alpha + 1}
                    \Comment{try the next value}
    \While{$\alpha \not\in \omega \And \omega \neq \bracesemptyset$}
                    \Comment{if this value is not in the domain}
      \Let {\alpha} {\alpha + 1}
                    \Comment{try the next value}
    \EndWhile
    \If{$\omega \neq \bracesemptyset$}
      \Let {\omega} {\omega \,\backslash \lbrace \alpha \rbrace}
                    \Comment{remove the value from the domain}
    \EndIf
    \State \Return $\alpha$
                    \Comment{\color{red} the next value (or the last value plus one if no more values)}
  \EndFunction
\end{algorithmic}
\end{cthalgorithm}

Now, let's rewrite \autoref{alg:advance1}, but now with a different customized visual, just for this entry!  The customization (3rd) argument is passed straight to the \texttt{tcolorbox} environment, so anything valid for \texttt{tcolorbox} is also valid here.

\begin{cthalgorithm}{Advance a counter to the next value in a domain $\omega \in \mathbb{N}$, but now with a customized visual. Also, notice that this algorithm breaks the page boundaries.}[coltitle=white, colback=green!10, colframe=green!70!black, fonttitle=\sffamily\bfseries\large, colbacktitle=red!50!white]
  \label{alg:advance2}
\begin{algorithmic}[1]
  \Statex \Input $\alpha$: The current value; and $\omega$: the domain.
  \Statex \Output The next value of $\alpha$ within the domain $\omega$.
  \LComment{\color{blue}
            e.g., $\alpha = 3 \And \omega = \lbrace 8, 3, 2, 21, 5, 1 \rbrace \rightarrow 5$}

  \Vspace

  \Function{advance}{$\alpha$: int, $\omega$: set} $\rightarrow$ int \textbf{is}
    % \Statex\Comment{As we are incrementing one by one on the operation IDs, the complexity of this function is linear in the value of the largest operation id.}
    \Let {\alpha} {\alpha + 1}
                    \Comment{try the next value}
    \While{$\alpha \not\in \omega \And \omega \neq \bracesemptyset$}
                    \Comment{if this value is not in the domain}
      \Let {\alpha} {\alpha + 1}
                    \Comment{try the next value}
    \EndWhile
    \If{$\omega \neq \bracesemptyset$}
      \Let {\omega} {\omega \,\backslash \lbrace \alpha \rbrace}
                    \Comment{remove the value from the domain}
    \EndIf
    \State \Return $\alpha$
                    \Comment{\color{red} the next value (or the last value plus one if no more values)}
  \EndFunction
\end{algorithmic}
\end{cthalgorithm}

Now, let's add an example here, this time using the default visual that was given when creating the environment with \verb+\cthnewtheorem+…

\begin{cthexample}{This is an example!}
  \emph{\kant[3]}
\end{cthexample}

And now an example with no caption and a different visual…  this example will not go into the \verb+\listofexamples+ below!

\begin{cthexample}{}[coltitle=yellow, colback=blue!30, colframe=blue!70]
  \emph{\kant[4]}
\end{cthexample}

Now let's print the lists of algorithms and examples. Remember to add the prefix \texttt{cthth} to the \texttt{listof}, i.e., \verb+\cthlistofalgorithms+ and \verb+\cthlistofexamples+!

\cthlistofalgorithms

\cthlistofexamples

\end{document}

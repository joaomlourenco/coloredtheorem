\documentclass{article}

%%%%%%%%%%%%%%%%%%%%%%%%%%%%%%%%%%%%%%%%%%
%% TCBTHEOREM stuff
\usepackage{coloredtheorem}

%%%%%%%%%%%%%%%%%%%%%%%%%%%%%%%%%%%%%%%%%%
%% EXAMPLE document
\usepackage[margin=1.6in]{geometry}
\usepackage{fancyvrb}
\usepackage{minted}
\usepackage{amsfonts}
\usepackage{kantlipsum}
\usepackage{tabularx}
\usepackage{xspace}
\newcommand\tcthrm{\textsf{\cthfilename}\xspace}

%% ALGPSEUDOCODEX stuff
\usepackage{algorithm}
\usepackage[
  noEnd=false,
  beginComment={$\triangleright$~\scriptsize},
  beginLComment={$\triangleright$~},
  endLComment={},
]{algpseudocodex}

\algnewcommand\Input{\textbf{Input:~}}
\algrenewcommand\Output{\textbf{Output:~}}
\algrenewcommand\algorithmicforall{\textbf{for each}}
\algnewcommand\Vspace{% 
  {\setlength\itemsep{-1ex}\item[]~}
}
\algnewcommand\Let[2]{\State \ensuremath{#1 \gets #2}}
\algnewcommand\And{\ensuremath{\wedge}\xspace}
\algnewcommand\Or{\ensuremath{\vee}\xspace}
\algnewcommand\Not{\ensuremath{\neg}\xspace}

\usepackage[colorlinks,linkcolor=blue!80]{hyperref}

%%%%%%%%%%%%%%%%%%%%%%%%%%%%%%%%%%%%%%%%%%
%% MAIN DOCUMENT
\begin{document}
  \title{The \tcthrm package}
  \author{João M. Lourenço\\\url{https://github.com/joaomlourenco/coloredtheorem}}
  \date{\cthfiledate\ (v\cthfileversion)}

  \maketitle
  

\section{Introduction}
\label{sec:introduction}

The \tcthrm package is a simple environment that allows to write stuff inside boxes from \texttt{tcolorbox}.  If necessary the boxes gracefully overflow to the next page.  This package takes no options and includes \texttt{tcolorbox} if necessary. You may include \texttt{tcolorbox} with your own favourite options prior to including this package.

Akin to \mintinline{TeX}+\newtheorem+ from the \texttt{amsmath} package, the user should start by defining a new \emph{theorem/box} group and customize its aspect. Each new environment will have its own counter/numbering. Notice that \mintinline{TeX}+\label{…}+ and \mintinline{TeX}+\ref{…}+ work as expected.  There is also a command to generate the corresponding \emph{list of …}

\section{Usage}
\label{sec:usage}

\definecolor{mycgreen}{RGB}{57,126,27}
\newcommand\ca[1]{\textcolor{blue}{#1}}
\newcommand\co[1]{\textcolor{blue!50}{#1}}
\newcommand\cc[1]{\textcolor{red!50!black}{#1}}
\newcommand\ck[1]{\textcolor{mycgreen}{#1}}

\begin{itemize}
  \item \mintinline{TeX}+\usepackage{coloredtheorem}+
  \begin{itemize}
      \item Load the \texttt{coloredtheorem} package.
            This package will load \texttt{tcolorbox} if necessary.
  \end{itemize}

  \item \mintinline[escapeinside=||]{TeX}+\cthnewtheorem{|\ca{<envname>}|}{|\ca{<Name>}|}[|\co{<tcolorbox options>}|]+
  \begin{itemize}
    \item Create a new boxed \emph{algorithm-like} environment.
    \begin{itemize}
      \item \mintinline[escapeinside=||]{TeX}+|\ca{<envname>}|+ is the suffix for the
            new environment being defined, e.g., \ca{\texttt{algoritm}}. The effective environment name will be \mintinline[escapeinside=||]{TeX}+|\ca{cth<envname>}|+, e.g., \ca{\texttt{cthalgorithm}}.
      \item \mintinline[escapeinside=||]{TeX}+|\ca{<Name>}|+ is the (printable) name
            or the new environment being defined, e.g., \emph{Algorithm}.
      \item \mintinline[escapeinside=||]{TeX}+|\co{<tcolorbox options>}|+ default
            options for the environment being defined (this argument is optional).
            These options are passed straight to the \texttt{tcolorbox} environment,
            so anything valid for \texttt{tcolorbox} is also valid here.
    \end{itemize}
  \end{itemize}

 \item \begin{minted}[escapeinside=||]{TeX}
\begin{|\ca{ctb<envname>}|}[|\co{<AltCaption>}|]{|\ca{<Caption>}|}[|\co{<tcolorbox options>}|]
  |\cc{<Contents>}|
\end{|\ca{ctb<envname>}|}
\end{minted}
  \begin{itemize}
    \item Create a new \emph{algorithm-like} box with the given contents.
    \begin{itemize}
      \item \mintinline[escapeinside=||]{TeX}+|\co{<AltCaption>}|+ is the alternative
            caption for the
            \mintinline[escapeinside=||]{TeX}+\cthlistof|\ck{<envname>s}|+
            (see below).
      \item \mintinline[escapeinside=||]{TeX}+|\ca{<envname>}|+ is the suffix for environment name,
            e.g., \verb+algorithm+.
      \item \mintinline[escapeinside=||]{TeX}+|\ca{<Caption>}|+ is the caption/title of
            the box.  If the caption is left empty, this box will not be listed with the
            \mintinline[escapeinside=||]{TeX}+\cthlistof|\ck{<envname>s}|+
            (see below).
      \item \mintinline[escapeinside=||]{TeX}+|\co{<tcolorbox options>}|+ options to be
            passed to the \verb+tcolorbox+ environment, which will override the defaults
            given in \mintinline{TeX}+\cthnewalgorithm+  (this argument is optional).
      \item \mintinline[escapeinside=||]{TeX}+|\cc{<Contents>}|+ the contents to by typeset inside the colored environment.
    \end{itemize}
  \end{itemize}

 \item \begin{minted}[escapeinside=||]{TeX}
\begin{|\ca{ctb<envname>*}|}{|\ca{<Caption>}|}[|\co{<tcolorbox options>}|]
  |\cc{<Contents>}|
\end{|\ca{ctb<envname>*}|}
\end{minted}
  \begin{itemize}
    \item Create a new \emph{algorithm-like} box with the given contents.
    \begin{itemize}
      \item \mintinline[escapeinside=||]{TeX}+|\ca{<envname>}|+ is the suffix for environment name,
            e.g., \verb+algorithm+.
      \item \mintinline[escapeinside=||]{TeX}+|\ca{<Caption>}|+ is the caption/title of
            the box.  This box will be unnumbered and will not be listed with the
            \mintinline[escapeinside=||]{TeX}+\cthlistof|\ck{<envname>s}|+
            (see below).
      \item \mintinline[escapeinside=||]{TeX}+|\co{<tcolorbox options>}|+ options to be
            passed to the \verb+tcolorbox+ environment, which will override the defaults
            given in \mintinline{TeX}+\cthnewalgorithm+  (this argument is optional).
      \item \mintinline[escapeinside=||]{TeX}+|\cc{<Contents>}|+ the contents to by typeset inside the colored environment.
    \end{itemize}
  \end{itemize}

  \item \mintinline[escapeinside=||]{TeX}+\cthlistof|\ck{<envname>s}|+
  \begin{itemize}
      \item \mintinline[escapeinside=||]{TeX}+|\ca{<envname>}|+ is the environment name
            suffix, e.g., \mintinline[escapeinside=||]{TeX}+\cthlistofalgorithms+.
            Please notice that there is a `s' (plural) after
            \mintinline[escapeinside=||]{TeX}+|\ca{<envname>}|+.
  \end{itemize}
\end{itemize}

\clearpage
\section{Example}
\label{sec:examples}

Let's start by creating two new environments, one for \emph{algorithms} and another for \emph{examples}, both defaulting to a gray frame, the former with a yellowish background and the latter with a lighter gray background.

\begin{minted}[bgcolor=red!5,fontsize=\small,escapeinside=||]{TeX}
\cthnewtheorem{|\ca{algorithm}|}{|\ca{Algorithm}|}[|\co{coltitle=}|black, |\co{colback=}|yellow!10,
                                     |\co{colframe=}|black!15]
\cthnewtheorem{|\ca{example}|}{|\ca{Example}|}[|\co{coltitle=}|black, |\co{colback}|=black!5,
                                 |\co{colframe=}|black!30]  
\end{minted}

\cthnewtheorem{algorithm}{Algorithm}[coltitle=black, colback=yellow!10, colframe=black!15]
\cthnewtheorem{example}{Example}[coltitle=black, colback=black!5, colframe=black!30]
\cthnewtheorem{thought}{Though}[enhanced, coltitle=green!60!black, colbacktitle=green!10, 
                                colback=green!10, frame hidden, boxrule=0pt, 
                                attach title to upper={\\[0.5ex]}, 
                                borderline west={1mm}{-2mm}{green!60!black}]

And this is a thought!
\begin{cththought}{This is a tought}
\kant[2]
\end{cththought}

The box with \autoref{alg:advance1}, which uses the default visual that was given when creating the environment with \mintinline{TeX}+\cthnewtheorem+ and gracefully overflows onto the next page, was created with:
\begin{minted}[bgcolor=red!5,fontsize=\small,escapeinside=||]{TeX}
\begin{|\ca{cthalgorithm}|}{|\co{Advance a counter to the next value in a domain}|
                     $\omega \in \mathbb{N}$.}
  |\cc{Algorithm body here!}|
\end{|\ca{cthalgorithm}|}
\end{minted}

\newcommand{\bracesemptyset}{\ensuremath{\lbrace\,\rbrace}}

\begin{cthalgorithm}{Advance a counter to the next value in a domain $\omega \in \mathbb{N}$.}
  \label{alg:advance1}
\begin{algorithmic}[1]
  \Statex \Input $\alpha$: The current value; and $\omega$: the domain.
  \Statex \Output The next value of $\alpha$ within the domain $\omega$.
  \LComment{\color{blue}
            e.g., $\alpha = 3 \And \omega = \lbrace 8, 3, 2, 21, 5, 1 \rbrace \rightarrow 5$}

  \Vspace

  \Function{advance}{$\alpha$: int, $\omega$: set} $\rightarrow$ int \textbf{is}
    % \Statex\Comment{As we are incrementing one by one on the operation IDs, the complexity of this function is linear in the value of the largest operation id.}
    \Let {\alpha} {\alpha + 1}
                    \Comment{try the next value}
    \While{$\alpha \not\in \omega \And \omega \neq \bracesemptyset$}
                    \Comment{if this value is not in the domain}
      \Let {\alpha} {\alpha + 1}
                    \Comment{try the next value}
    \EndWhile
    \If{$\omega \neq \bracesemptyset$}
      \Let {\omega} {\omega \,\backslash \lbrace \alpha \rbrace}
                    \Comment{remove the value from the domain}
    \EndIf
    \State \Return $\alpha$
                    \Comment{\color{red} the next value (or the last value plus one if no more values)}
  \EndFunction
\end{algorithmic}
\end{cthalgorithm}

The box \autoref{expl:explstd}, which uses the default visual for the environment (as given to \mintinline{TeX}+\cthnewtheorem...+), was created with:

\begin{minted}[bgcolor=red!5,fontsize=\small,escapeinside=||]{TeX}
\begin{|\ca{cthexample}|}{|\co{This is an example!}|}
  |\cc{Example body here!}|
\end{|\ca{cthexample}|}
\end{minted}


\begin{cthexample}{This is an example!}
  \label{expl:explstd}
  \emph{As is shown in the writings of Aristotle, the things in themselves (and it remains a mystery why this is the case) are a representation of time. Our concepts have lying before them the paralogisms of natural reason, but our a posteriori concepts have lying before them the practical emploument of our experience.}
\end{cthexample}

Now, let's create a new box for \autoref{alg:advance1}, but with a different customized visual, which will affect only this entry!  Notice that the customization argument is passed straight to the \texttt{tcolorbox} environment, so anything valid for \texttt{tcolorbox} is also valid here.  \autoref{alg:advance2} was created with the following code:

\begin{minted}[bgcolor=red!5,fontsize=\small,escapeinside=||]{TeX}
\begin{|\ca{cthalgorithm}|}
          {|\ca{Advance a counter to the next value in a domain}| $\omega \in
                \mathbb{N}$|\ca{, but now with a customized visual. Also, notice}|
                |\ca{that this algorithm breaks the page boundaries.}|}
          [|\co{coltitle}=|white, |\co{colback}=|green!10, |\co{colframe}=|green!70!black,
                |\co{colbacktitle}=|\sffamily\bfseries\large, |\co{fonttitle}|=red!50!white]
  |\cc{Algorithm body here!}|
\end{|\ca{cthalgorithm}|}
\end{minted}

\begin{cthalgorithm}{Advance a counter to the next value in a domain $\omega \in \mathbb{N}$, but now with a customized visual. Notice that this algorithm also breaks the page boundaries.}[coltitle=white, colback=red!10, colframe=red!50!black, fonttitle=\sffamily\bfseries\large, colbacktitle=red!70!black]
  \label{alg:advance2}
\begin{algorithmic}[1]
  \Statex \Input $\alpha$: The current value; and $\omega$: the domain.
  \Statex \Output The next value of $\alpha$ within the domain $\omega$.
  \LComment{\color{blue}
            e.g., $\alpha = 3 \And \omega = \lbrace 8, 3, 2, 21, 5, 1 \rbrace \rightarrow 5$}

  \Vspace

  \Function{advance}{$\alpha$: int, $\omega$: set} $\rightarrow$ int \textbf{is}
    % \Statex\Comment{As we are incrementing one by one on the operation IDs, the complexity of this function is linear in the value of the largest operation id.}
    \Let {\alpha} {\alpha + 1}
                    \Comment{try the next value}
    \While{$\alpha \not\in \omega \And \omega \neq \bracesemptyset$}
                    \Comment{if this value is not in the domain}
      \Let {\alpha} {\alpha + 1}
                    \Comment{try the next value}
    \EndWhile
    \If{$\omega \neq \bracesemptyset$}
      \Let {\omega} {\omega \,\backslash \lbrace \alpha \rbrace}
                    \Comment{remove the value from the domain}
    \EndIf
    \State \Return $\alpha$
                    \Comment{\color{red} the next value (or the last value plus one if no more values)}
  \EndFunction
\end{algorithmic}
\end{cthalgorithm}

And now an example with no caption and a different visual…  this example will not go into the \mintinline{TeX}+\listofexamples+ below!

\begin{cthexample}{}[coltitle=black, colback=magenta!10, colframe=magenta!20]
  \emph{\kant[4]}
\end{cthexample}

And another unnumbered example using an alternative syntax:


\begin{cthexample*}{This is an unnumbered example with a title}[coltitle=black, colback=yellow!10, colframe=yellow!20]
  \emph{\kant[4]}
\end{cthexample*}

Now let's print the lists of algorithms and examples. Remember to add the prefix \ck{\texttt{cth}} to the \ck{\texttt{listof}}, i.e., \mintinline{TeX}+\cthlistofalgorithms+ and \mintinline{TeX}+\cthlistofexamples+!

\cthlistofalgorithms

\cthlistofexamples

\end{document}
